\documentclass[../lecture-notes-148x210.tex]{subfiles}

\begin{document}
\subsection{Introduction}

\textbf{Bulletproofs} is a non-interactive zero-knowledge proof protocol with logarithmically sized proofs without a trusted setup. Originally, \textbf{bulletproofs} was developed to provide efficient inner-product proofs for range proofs in application to confidential transactions, but it applies to arbitrary arithmetic circuit (possibly encoded in R1CS). Technically, the protocol is built in an interactive fashion (like $\Sigma$-protocols) and made non-interactive with a Fiat-Shamir transform. One key feature that differs it from $\Sigma$-protocols is the number of challenges from a verifier $\mathcal{V}$: in $\Sigma$-protocols there is only one challenge, while \textbf{bulletproofs} implies a logarithmic in circuit size number of queries. The main advantages of \textbf{bulletproofs} are an absence of a trusted setup, security that relies on the \textit{discrete-logarithm} assumption without any other auxiliary structures like bilinear pairings, however, the main disadvantage of \textbf{bulletproofs} is linear in circuit size verification time though still efiicient for small circuits.

Hence, we will describe some preliminaries.

\subsection{Proving non-linear relations with $\Sigma$-protocols}

\subsection{Inner-product argument}

Firstly, we describe the protocol for the \textbf{inner-product argument} - core component of the \textbf{bulletproofs} protocol. After that we will apply it to range proofs and arithmetic circuits. We have already seen that inner-products are the main ingridients for R1CS language because any R1CS relation could be seen as a batch of inner-products though it's not the most efficient representation and we'll see how to amortize all the constraints into inner-products more efficiently.

Let $\mathbb{G}$ - cyclic group of prime order $p$ written additively, $\mathbf{G} = (G_1, \dots, G_n), \mathbf{H} = (H_1, \dots, H_n) \in \mathbb{G}^n$ - vectors of independent generators. We denote by $\langle \mathbf{a,b} \rangle$ - inner product of vectors $\mathbf{a} = (a_1, \dots, a_n), \mathbf{b} = (b_1, \dots, b_n) \in \mathbb{F}_p^n$ and $\langle \mathbf{a,G} \rangle = \sum_{i=1}^n [a_i] G_i \in \mathbb{G}$ - inner product of vector $\mathbf{a}$ with vector of generators $\mathbf{G}$.

The \textbf{inner-product argument} allows to prove that two vectors $\mathbf{a,b} \in \mathbb{F}_p^n$ satisfy the relation:
$$\mathcal{R}_{ip} = \{ (\mathbf{G,H}, P, c; \mathbf{a,b}) \vert P = \langle \mathbf{a,G} \rangle + \langle \mathbf{b,H} \rangle \wedge \langle \mathbf{a,b} \rangle = c \}$$
We refer to $P \in \mathbb{G}$ as a binding Pedersen vector commitment to $\mathbf{a,b}$.

Trivial way to prove the relation is to send $\mathbf{a,b}$ to the verifier $\mathcal{V}$, but it is not a zero-knowledge proof nor efficient due to linear in $n$ size of the proof. We want to build a zero-knowledge proof of the relation $\mathcal{R}_{ip}$ with logarithmic in $n$ size of the proof.

Firtsly, let's combine statements $P = \langle \mathbf{a,G} \rangle + \langle \mathbf{b,H} \rangle \wedge \langle \mathbf{a,b} \rangle = c$ into a single statement by multiplying the second one by a random $r \in \mathbb{F}_p$ and some orthogonal generator $B \in \mathbb{G}$, summing up:
$$
\mathcal{R}'_{ip} = \{ (\mathbf{G,H}, Q, P'; \mathbf{a,b}) \vert P' = \langle \mathbf{a,G} \rangle + \langle \mathbf{b,H} \rangle + [\langle \mathbf{a,b} \rangle]Q \}
$$
Where $P' = P + [cr]B, Q=[r]B$. Intuitively, if prover $\mathcal{P}$ can prove $\mathcal{R}'_{ip}$ for all $r \in \mathbb{F}_p$, then it can prove $\mathcal{R}_{ip}$ for any valid witness. We use such transformation to compress each vector in half and arrive to the same form of commitment
$$P' = \langle \mathbf{a,G} \rangle + \langle \mathbf{b,H} \rangle + [\langle \mathbf{a,b} \rangle]Q$$

\begin{definition}
    The \textbf{inner-product} protocol $\Pi_{ip} = (\mathcal{P}, \mathcal{V})$ for relation $\mathcal{R}_{ip} = \{ (\mathbf{G,H}, P, c; \mathbf{a,b}) \vert P = \langle \mathbf{a,G} \rangle + \langle \mathbf{b,H} \rangle \wedge \langle \mathbf{a,b} \rangle = c \}$ with prover $\mathcal{P}$, verifier $\mathcal{V}$ is defined as follows:
    \begin{itemize}
        \item Parties $\mathcal{P, V}$ agree on some group element $B \in \mathbb{G}$ with unknown discrete log
        \item Verifier $\mathcal{V}$ samples random $r \xleftarrow{R} \mathbb{F}_p$ and sends it to prover $\mathcal{P}$
        \item Parties $\mathcal{P, V}$ compute $Q \gets [r]B$ and $P' = P + [c]Q$
        \item Parties run protocol $\Pi'_{ip}$ for relation $\mathcal{R}'_{ip}$ on input $(\mathbf{G,H}, Q, P'; \mathbf{a,b})$
    \end{itemize}
\end{definition}

\subsubsection{Inner-product compression}

Here we describe \textbf{inner-product compression} algorithm -- main building block of the interactive \textbf{inner-product} procotol.
Firstly, assuming that $n = 2^d$ define by $\mathbf{G_{lo}} = (G_1, \dots, G_{n/2}), \mathbf{G_{hi}} = (G_{n/2+1},\dots, G_n) \in \mathbb{G}^{n/2}$ -- lower and higher halves of vector $\mathbf{G}$ and $\mathbf{a_{lo}} = (a_1, \dots, a_{n/2}), \mathbf{a_{hi}} = (a_{n/2+1},\dots,a_n) \in \mathbb{F}_{n/2}$ -- lower and higher halves of $\mathbf{a} \in \mathbb{F}_p^{n}$.

Let $u_k \in \mathbb{F}_p$ - be some scalar, define compressed vectors:
\begin{align*}
    \mathbf{a}^{(k-1)} &= \mathbf{a_{lo}} \cdot u_k + u_k^{-1} \cdot \mathbf{a_{hi}} \\
    \mathbf{b}^{(k-1)} &= \mathbf{b_{lo}} \cdot u_k^{-1} + u_k \cdot \mathbf{b_{hi}} \\
    \mathbf{G}^{(k-1)} &= \mathbf{G_{lo}} \cdot u_k^{-1} + u_k \cdot \mathbf{G_{hi}} \\
    \mathbf{H}^{(k-1)} &= \mathbf{H_{lo}} \cdot u_k + u_k^{-1} \cdot \mathbf{H_{hi}}
\end{align*}

Define $P_k \gets P' =  \langle \mathbf{a,G} \rangle + \langle \mathbf{b,H} \rangle + [\langle \mathbf{a,b} \rangle]Q$ and define $P_{k-1}$ using compressed vectors to have the same form as $P_k$:
$$P_{k-1} = \langle \mathbf{a}^{(k-1)}, \mathbf{G}^{(k-1)} \rangle + \langle \mathbf{b}^{(k-1)}, \mathbf{H}^{(k-1)} \rangle + [\langle \mathbf{a}^{(k-1)}, \mathbf{b}^{(k-1)} \rangle]Q $$

Subsituting compressed vectors and applying bilinearity property of inner product we get:
\begin{align*}
    P_{k-1} = & \langle \mathbf{a_{lo}}, \mathbf{G_{lo}}\rangle + \langle \mathbf{a_{hi}}, \mathbf{G_{hi}}\rangle &+ u_k^2\langle \mathbf{a_{lo}}, \mathbf{G_{hi}}\rangle + u_k^{-2}\langle \mathbf{a_{hi}}, \mathbf{G_{lo}}\rangle + \\
    & \langle \mathbf{b_{lo}}, \mathbf{H_{lo}}\rangle + \langle \mathbf{b_{hi}}, \mathbf{H_{hi}}\rangle &+ u_k^2\langle \mathbf{b_{hi}}, \mathbf{H_{lo}}\rangle + u_k^{-2}\langle \mathbf{b_{lo}}, \mathbf{H_{hi}}\rangle + \\
    & [\langle \mathbf{a_{lo}}, \mathbf{b_{lo}}\rangle + \langle \mathbf{a_{hi}}, \mathbf{b_{hi}}\rangle]Q &+ [u_k^2\langle \mathbf{a_{lo}}, \mathbf{b_{hi}}\rangle + u_k^{-2}\langle \mathbf{a_{hi}}, \mathbf{b_{lo}}\rangle]Q
\end{align*}

Note that $\langle \mathbf{a_{lo}}, \mathbf{G_{lo}}\rangle + \langle \mathbf{a_{hi}}, \mathbf{G_{hi}}\rangle = \langle \mathbf{a,G}\rangle$ so that the first two columns of $P_{k-1}$ definition precisecly contains $P_{k} = P'$:
$$P_{k} = \langle \mathbf{a_{lo}}, \mathbf{G_{lo}}\rangle + \langle \mathbf{a_{hi}}, \mathbf{G_{hi}}\rangle + \langle \mathbf{b_{lo}}, \mathbf{H_{lo}}\rangle + \langle \mathbf{b_{hi}}, \mathbf{H_{hi}}\rangle + [\langle \mathbf{a_{lo}}, \mathbf{b_{lo}}\rangle + \langle \mathbf{a_{hi}}, \mathbf{b_{hi}}\rangle]Q$$

Define cross-terms $L_k, R_k$ of $P_{k-1}$ such that:
\begin{align*}
    P_{k-1} &= P_k + [u_k^2] L_k + [u_k^{-2}] R_k \\
    L_{k} &= \langle \mathbf{a_{lo}}, \mathbf{G_{hi}}\rangle + \langle \mathbf{b_{hi}}, \mathbf{H_{lo}}\rangle + [\langle \mathbf{a_{lo}}, \mathbf{b_{hi}}\rangle]Q \\
    R_{k} &= \langle \mathbf{a_{hi}}, \mathbf{G_{lo}}\rangle + \langle \mathbf{b_{lo}}, \mathbf{H_{hi}}\rangle + [\langle \mathbf{a_{hi}}, \mathbf{b_{lo}}\rangle]Q
\end{align*}

Here we come up with some kind of statement compression algorithm reducing size of all vectors in half per compression step. Repeating comression algorithm $k$ times we end up with vectors $\mathbf{a}^{(0)}, \mathbf{b}^{(0)}, \mathbf{G}^{(0)}, \mathbf{H}^{(0)}$ each of length one and $P_0$ containing all accumulated cross-terms:
\begin{align*}
    P_0 &= [a_1^{(0)}]G_1^{(0)} + [b_1^{(0)}]H_1^{(0)} + [a_1^{(0)}b_1^{(0)}]Q \\
    P_0 &= P_k + \sum_{i=1}^d ([u_i^2]L_i + [u_i^{-2}]R_i)
\end{align*}

Recalling that $P_k = P'$, the final compressed statement asserting inner-product value will have the following form:
\begin{equation}
    P' + \sum_{i=1}^d ([u_i^2]L_i + [u_i^{-2}]R_i) = [a_1^{(0)}]G_1^{(0)} + [b_1^{(0)}]H_1^{(0)} + [a_1^{(0)}b_1^{(0)}]Q
    \label{eq:ip-final-compressed}
\end{equation}

\subsubsection{Proving $\mathcal{R}'_{ip}$}

Let's describe the \textbf{inner-product} protocol $\Pi'_{ip}$ for relation $\mathcal{R}'_{ip}$. 
\begin{definition}
    The \textbf{inner-product} protocol $\Pi'_{ip} = (\mathcal{P}, \mathcal{V})$ for relation $\mathcal{R}'_{ip} = \{ (\mathbf{G,H}, Q, P'; \mathbf{a,b}) \vert P' = \langle \mathbf{a,G} \rangle + \langle \mathbf{b,H} \rangle + [\langle \mathbf{a,b} \rangle]Q \}$, where all vectors have length $n=2^d$ with prover $\mathcal{P}$, verifier $\mathcal{V}$ is defined as follows:
    \begin{itemize}
        \item Prover $\mathcal{P}$ sets $$(k, \mathbf{a}^{(k)}, \mathbf{b}^{(k)}, \mathbf{G}^{(k)}, \mathbf{H}^{(k)}, P_k) \gets (n, \mathbf{a,b,G,H},P')$$
        \item Verifier $\mathcal{V}$ sets $$(k, \mathbf{G}^{(k)}, \mathbf{H}^{(k)}, P_k) \gets (n, \mathbf{G,H},P')$$
        \item While $k > 0$ then:
            \begin{itemize}
                \item Prover $\mathcal{P}$ computes 
                    \begin{align*}
                        L_{k} &= \langle \mathbf{a_{lo}}^{(k)}, \mathbf{G_{hi}}^{(k)}\rangle + \langle \mathbf{b_{hi}}^{(k)}, \mathbf{H_{lo}}^{(k)}\rangle + [\langle \mathbf{a_{lo}}^{(k)}, \mathbf{b_{hi}}^{(k)}\rangle]Q \\
                        R_{k} &= \langle \mathbf{a_{hi}}^{(k)}, \mathbf{G_{lo}}^{(k)}\rangle + \langle \mathbf{b_{lo}}^{(k)}, \mathbf{H_{hi}}^{(k)}\rangle + [\langle \mathbf{a_{hi}}^{(k)}, \mathbf{b_{lo}}^{(k)}\rangle]Q
                    \end{align*}
                \item $\mathcal{P}$ sends $(L_k, R_k)$ to $\mathcal{V}$
                \item $\mathcal{V}$ draws challenge $u_k \xleftarrow{R} \mathbb{F}_p$ and sends it to $\mathcal{P}$
                \item Both $\mathcal{P}$ and $\mathcal{V}$ compute:
                    \begin{align*}
                        \mathbf{G}^{(k-1)} &= \mathbf{G_{lo}}^{(k)} \cdot u_k^{-1} + u_k \cdot \mathbf{G_{hi}}^{(k)} \\
                        \mathbf{H}^{(k-1)} &= \mathbf{H_{lo}}^{(k)} \cdot u_k + u_k^{-1} \cdot \mathbf{H_{hi}}^{(k)}
                    \end{align*}
                \item $\mathcal{P}$ computes:
                    \begin{align*}
                        \mathbf{a}^{(k-1)} &= \mathbf{a_{lo}}^{(k)} \cdot u_k + u_k^{-1} \cdot \mathbf{a_{hi}}^{(k)} \\
                        \mathbf{b}^{(k-1)} &= \mathbf{b_{lo}}^{(k)} \cdot u_k^{-1} + u_k \cdot \mathbf{b_{hi}}^{(k)} \\
                    \end{align*}
            \end{itemize}
        \item Prover $\mathcal{P}$ sends $(a,b) \gets (\mathbf{a}_1^{(0)}, \mathbf{b}_1^{(0)})$ to verifier $\mathcal{V}$
        \item Verifier performs final check: 
            $$P' + \sum_{i=1}^d ([u_i^2]L_i + [u_i^{-2}]R_i) = [a]G_1^{(0)} + [b]H_1^{(0)} + [ab]Q$$
        outputs \textbf{accept} if equality holds and \textbf{reject} otherwise.
    \end{itemize}

    This protocol is illustrated in \Cref{fig:interactive_ip}. 
\end{definition}

\begin{figure}[h!]
    \centering
    % TikZ inner-product protocol diagram with highlighted loop
    \begin{tikzpicture}[>=Stealth,thick,node distance=5cm,auto]
        \usetikzlibrary{fit,backgrounds,positioning}
        % Prover and Verifier nodes
        \node[inner sep=0pt] (P) {\begin{tabular}{c}\includegraphics[width=1.2cm]{lectures/images/common/prover.png}\\Prover $\mathcal{P}$\end{tabular}};
        \node[inner sep=0pt,right=of P] (V) {\begin{tabular}{c}\includegraphics[width=1.2cm]{lectures/images/common/verifier.png}\\Verifier $\mathcal{V}$\end{tabular}};
        % Dashed lifelines extended
        \draw[dashed,line width=0.3mm] ([yshift=-0cm]P.south) -- ++(0,-12cm);
        \draw[dashed,line width=0.3mm] ([yshift=-0cm]V.south) -- ++(0,-12cm);

        % Step 1: Begin loop
        \node[coordinate] (Ploopstart) at ([yshift=-1cm]P.south) {};
        \node[coordinate] (Vloopstart) at ([yshift=-1cm]V.south) {};
        
        \node[draw,rounded corners,fill=white,below=1cm of P,align=left] (LR) {%
            \small $L_{k} \gets \langle \mathbf{a_{lo}}^{(k)}, \mathbf{G_{hi}}^{(k)}\rangle +  $\\
            \small $\langle \mathbf{b_{hi}}^{(k)}, \mathbf{H_{lo}}^{(k)}\rangle + [\langle \mathbf{a_{lo}}^{(k)}, \mathbf{b_{hi}}^{(k)}\rangle]Q $ \\
            \small $R_{k} \gets \langle \mathbf{a_{hi}}^{(k)}, \mathbf{G_{lo}}^{(k)}\rangle +  $\\
            \small $\langle \mathbf{b_{lo}}^{(k)}, \mathbf{H_{hi}}^{(k)}\rangle + [\langle \mathbf{a_{hi}}^{(k)}, \mathbf{b_{lo}}^{(k)}\rangle]Q$
        };

        % Step 2: Prover sends L_k, R_k
        \node[coordinate] (Vloop1) at ([yshift=-3.5cm]V.south) {};
        \node[coordinate] (Ploop1) at ([yshift=-3.5cm]P.south) {};
        \draw[->] (Ploop1) -- node[above]{Send $(L_k,R_k)$} (Vloop1);

        % Step 3: Challenge
        \node[draw,rounded corners,fill=white,below=4cm of V,align=left] (pchelV) {%
            \small $u_k \xleftarrow{R} \mathbb{F}_p$
        };


        % Step 2: Verifier sends u_k
        \node[coordinate] (Vloop2) at ([yshift=-5cm]V.south) {};
        \node[coordinate] (Ploop2) at ([yshift=-5cm]P.south) {};
        \draw[->] (Vloop2) -- node[above]{Send $u_k$} (Ploop2);

        % Step 3: Updates
        \node[draw,rounded corners,fill=white,below=5.5cm of P,align=left] (updatesP) {%
            \small $\mathbf{G}^{(k-1)}\gets \mathbf{G}_{lo}^{(k)}u_k^{-1}+u_k\mathbf{G}_{hi}^{(k)}$\\
            \small $\mathbf{H}^{(k-1)}\gets \mathbf{H}_{lo}^{(k)}u_k+u_k^{-1}\mathbf{H}_{hi}^{(k)}$\\
            \small $\mathbf{a}^{(k-1)}\gets \mathbf{a}_{lo}^{(k)}u_k+u_k^{-1}\mathbf{a}_{hi}^{(k)}$\\
            \small $\mathbf{b}^{(k-1)}\gets \mathbf{b}_{lo}^{(k)}u_k^{-1}+u_k\mathbf{b}_{hi}^{(k)}$
        };

        \node[draw,rounded corners,fill=white,below=5.5cm of V,align=left] (updatesV) {%
            \small $\mathbf{G}^{(k-1)}\gets \mathbf{G}_{lo}^{(k)}u_k^{-1}+u_k\mathbf{G}_{hi}^{(k)}$\\
            \small $\mathbf{H}^{(k-1)}\gets \mathbf{H}_{lo}^{(k)}u_k+u_k^{-1}\mathbf{H}_{hi}^{(k)}$
        };

        % Loop region (highlighted)
        \begin{scope}[on background layer]
        \node[draw,rounded corners,fill=blue!10,inner sep=6pt,fit=(Ploopstart)(Vloopstart)(updatesP)(updatesV)(LR)] (loopBlock) {};
        \end{scope}
        \node[above=0cm of loopBlock.north] {\small \textbf{Repeat for }$k \gets \overline{d..1}$};

        % Final step: Prover sends (a,b)
        \node[coordinate] (Pfinal) at ([yshift=-9cm]P.south) {};
        \node[coordinate] (Vfinal) at ([yshift=-9cm]V.south) {};
        \draw[->] (Pfinal) -- node[above]{Send $(a,b) = (\mathbf{a}^{(0)},\mathbf{b}^{(0)})$} (Vfinal);

        % Verifier final check
        \node[draw,rounded corners,fill=white,below=0.5cm of Vfinal,align=left] (checkV) {%
            \small Verify:\\
            $P'+\sum_{i=1}^d([u_i^2]L_i+[u_i^{-2}]R_i)=$\\
            $[a]G_1^{(0)}+[b]H_1^{(0)}+[ab]Q$
        };
    \end{tikzpicture}
    \caption{Interactive inner-product protocol $\Pi'_{ip}$ between prover $\mathcal{P}$ and verifier $\mathcal{V}$ for relation $\mathcal{R}'_{ip}$}
    \label{fig:interactive_ip}
\end{figure}

As we can see, overall communication complexity of $\Pi_{ip}$ is $2\log_2 n \mathbb{G} + 2 \mathbb{F}_p$ so we come up with logarithmic proof size for our inner-product relation $\mathcal{R}_{ip}$. Now one might ask whether $\Pi_{ip}$ has any desired properties such as completeness, soundness and zero-knowledge. It appears that the first two holds under some generalizations needed for security proofs but not zero-knowledge (indeed, if $n=1$ then $\mathcal{P}$ sends witness pair $a,b$ directly).

\begin{theorem}[Inner-Product Argument]
    The argument system $\Pi_{ip}$ for relation $\mathcal{R}_{ip}$ has \textit{perfect completeness and statistical witness-extended emulation} for either extracting a non-trivial discrete logarithm relation between $\mathbf{G,H}, Q$ or extracting valid witness $\mathbf{a,b}$.
\end{theorem}

\textbf{Proof idea}. \textit{Perfect completeness} of $Pi_{ip}$ follows because $Pi_{ip}$ converts instance of $\mathcal{R}_{ip}$ to instance of $\mathcal{P}'_{ip}$ and $\Pi'_{ip}$ is trivially complete by construction due to \Cref{eq:ip-final-compressed}. Notation \textit{statistical witness-extended emulation} generalizes special soundness in the way applicable for multi-stage complex argument systems where each step of the protocol could be rewinded to extract part of the witness so that more accurate definition of protocol security is achieved despite \textit{special soundness} implies building the whole knowledge extractor which might has non-polynomial running time for multi-stage protocols.

Here we briefly describe a knowledge extractor $\mathcal{E}$ for a witness $(\mathbf{a}^{(k)}, \mathbf{b}^{(k)})$ or non-trivial discrete logarithm relation for $(\mathbf{G, H}, Q)$ for one stage of protocol. 
\begin{enumerate}
    \item $\mathcal{E}$ runs $\mathcal{P}$ to the last stage to obtain $(L, R) \gets (L_1, R_1)$
    \item $\mathcal{E}$ rewinds the $\Pi'_{ip}$ last stage four times to obtain four challenges $x_1, x_2, x_3, x_4$, so that the following equations holds from expressing $P_{1}$:
    \begin{align}
        [x_1^2] L_k + P_k + [x_1^{-2}] R_k = \langle \mathbf{a}^{(k-1)}, \mathbf{G}^{(k-1)} \rangle + \langle \mathbf{b}^{(k-1)}, \mathbf{H}^{(k-1)} \rangle + [\langle \mathbf{a}^{(k-1)}, \mathbf{b}^{(k-1)} \rangle]Q
    \end{align}
    \item 
\end{enumerate}
\subsection{Range proofs}

\subsection{Arithmetic circuits proofs}



\subsection*{Acknowledgements}

\begin{itemize}
    \item \href{https://doc-internal.dalek.rs/bulletproofs/index.html}{dalek's crate bulletproofs}
    \item \href{https://eprint.iacr.org/2017/1066.pdf}{bulletproofs whitepaper}
\end{itemize}

\end{document}
