\documentclass[../lecture-notes-148x210.tex]{subfiles}

\begin{document}
\subsection{Introduction}

\textbf{Bulletproofs} is a non-interactive zero-knowledge proof protocol with logarithmically sized proofs without a trusted setup. Originally, \textbf{bulletproofs} was developed to provide efficient inner-product proofs for range proofs in application to confidential transactions, but it applies to arbitrary arithmetic circuit (possibly encoded in R1CS). Technically, the protocol is built in an interactive fashion (like $\Sigma$-protocols) and made non-interactive with a Fiat-Shamir transform. One key feature that differs it from $\Sigma$-protocols is the number of challenges from a verifier $\mathcal{V}$: in $\Sigma$-protocols there is only one challenge, while \textbf{bulletproofs} implies a logarithmic in circuit size number of queries. The main advantages of \textbf{bulletproofs} are an absence of a trusted setup, security that relies on the \textit{discrete-logarithm} assumption without any other auxiliary structures like bilinear pairings, however, the main disadvantage of \textbf{bulletproofs} is linear in circuit size verification time though still efiicient for small circuits.

Hence, we will describe some preliminaries.

\subsection{Proving non-linear relations with $\Sigma$-protocols}

\subsection{Inner-product argument}

Firstly, we describe the protocol for the \textbf{inner-product argument} - core component of the \textbf{bulletproofs} protocol. After that we will apply it to range proofs and arithmetic circuits. We have already seen that inner-products are the main ingridients for R1CS language because any R1CS relation could be seen as a batch of inner-products though it's not the most efficient representation and we'll see how to amortize all the constraints into inner-products more efficiently.

Let $\mathbb{G}$ - cyclic group of prime order $p$ written additively, $\mathbf{G} = (G_1, \dots, G_n), \mathbf{H} = (H_1, \dots, H_n) \in \mathbb{G}^n$ - vectors of independent generators. We denote by $\langle \mathbf{a,b} \rangle$ - inner product of vectors $\mathbf{a} = (a_1, \dots, a_n), \mathbf{b} = (b_1, \dots, b_n) \in \mathbb{F}_p^n$ and $\langle \mathbf{a,G} \rangle = \sum_{i=1}^n [a_i] G_i \in \mathbb{G}$ - inner product of vector $\mathbf{a}$ with vector of generators $\mathbf{G}$.

The \textbf{inner-product argument} allows to prove that two vectors $\mathbf{a,b} \in \mathbb{F}_p^n$ satisfy the relation:
$$\mathcal{R}_{ip} = \{ (\mathbf{G,H}, P, c; \mathbf{a,b}) \vert P = \langle \mathbf{a,G} \rangle + \langle \mathbf{b,H} \rangle \wedge \langle \mathbf{a,b} \rangle = c \}$$
We refer to $P \in \mathbb{G}$ as a binding Pedersen vector commitment to $\mathbf{a,b}$.

Trivial way to prove the relation is to send $\mathbf{a,b}$ to the verifier $\mathcal{V}$, but it is not a zero-knowledge proof nor efficient due to linear in $n$ size of the proof. We want to build a zero-knowledge proof of the relation $\mathcal{R}_{ip}$ with logarithmic in $n$ size of the proof.

Firtsly, let's combine statements $P = \langle \mathbf{a,G} \rangle + \langle \mathbf{b,H} \rangle \wedge \langle \mathbf{a,b} \rangle = c$ into a single statement by multiplying the second one by a random $r \in \mathbb{F}_p$ and some orthogonal generator $B \in \mathbb{G}$, summing up:
$$
\mathcal{R}'_{ip} = \{ (\mathbf{G,H}, P', c; \mathbf{a,b}) \vert P' = \langle \mathbf{a,G} \rangle + \langle \mathbf{b,H} \rangle + [\langle \mathbf{a,b} \rangle]Q \}
$$
Where $P' = P + [cr]B, Q=[r]B$. Intuitively, if prover $\mathcal{P}$ can prove $\mathcal{R}'_{ip}$ for all $r \in \mathbb{F}_p$, then it can prove $\mathcal{R}_{ip}$ for any valid witness. We use such transformation to compress each vector in half and arrive to the same form of commitment
$$P' = \langle \mathbf{a,G} \rangle + \langle \mathbf{b,H} \rangle + [\langle \mathbf{a,b} \rangle]Q$$

\subsubsection{Proving $\mathcal{R}'_{ip}$}

Here we describe the \textbf{inner-product argument} protocol $\Pi_{ip}$ for the relation $\mathcal{R}'_{ip}$. 
Firstly, assuming that $n = 2^d$ define by $\mathbf{G_{lo}, G_{hi}} \in \mathbb{G}^{n/2}$ -- lower and higher halves of vector $\mathbf{G}$ and $\mathbf{a_{lo}, a_{hi}} \in \mathbb{F}_{n/2}$ -- lower and higher halves of $\mathbf{a} \in \mathbb{F}_p^{n}$.

Let $u_k \in \mathbb{F}_p$ - be some scalar, define compressed vectors:
\begin{align}
\mathbf{a}^{(k-1)} &= \mathbf{a}_{lo} \cdot u_k + u_k^{-1} \cdot \mathbf{a}_{hi} \\
\mathbf{b}^{(k-1)} &= \mathbf{b}_{lo} \cdot u_k^{-1} + u_k \cdot \mathbf{b}_{hi} \\
\mathbf{G}^{(k-1)} &= \mathbf{G}_{lo} \cdot u_k^{-1} + u_k \cdot \mathbf{G}_{hi} \\
\mathbf{H}^{(k-1)} &= \mathbf{H}_{lo} \cdot u_k + u_k^{-1} \cdot \mathbf{H}_{hi}
\end{align}

Define $P_k \gets P'$ and define $P_{k-1}$ using compressed vectors to have the same form as $P_k$:
$$P_{k-1} = \langle \mathbf{a}^{(k-1)}, \mathbf{G}^{(k-1)} \rangle + \langle \mathbf{b}^{(k-1)}, \mathbf{H}^{(k-1)} \rangle + [\langle \mathbf{a}^{(k-1)}, \mathbf{b}^{(k-1)} \rangle]Q $$

\subsection{STARK-friendly fields}
In general, STARK protocol can work over any field $\mathbb{F}$ with high two-adicity. The primary 
reason for that is that STARKs can work only with NTT-friendly fields, and the
NTT-friendly fields are the fields where we can select the multiplicative
subgroup of order $2^k$ for sufficiently many values of $k$. 

\begin{definition}
    We call \textbf{two-adicity fields}, the fields where we can select the
    multiplicative subgroup of order $2^k$ for sufficiently many values of $k$.
    In this case, the field order $p$ is typically of form $p = 2^m \cdot p' +
    1$ where $p'$ is a small integer.
\end{definition}

To be honest, all protocol steps are followed with powers of two. It will be shown, why the groups we are working over must be of size $2^k$ and why the input data also follows this rule. As the result, the maximum size of the statement that we can prove using the STARK protocol is strictly depends on the size of two-adicity subgroup (that is why we label some fields to have \textit{high two-adicity} or \textit{low two-adicity}).

\begin{remark}
In our initial discussion we consider using field over prime modulus $p = 3\cdot
2^{30} + 1$ and subgroups of size $2^{13}$ and $2^{10}$.
\end{remark}

As we will work in the new subgroup we may want to specify the subgroup
generator to be used in future equations. So, for the multiplicative group
generator $w \in \mathbb{F}_p^{\times}$, the generator of the subgroup of order
$2^k$ is $\omega_k = w^\frac{p - 1}{2^k}$, as was shown in the NTT section.

\begin{example}
For the prime field $\mathbb{F}_p$ where $p = 3\cdot 2^{30} + 1$, the order of $\mathbb{F}^{\times}_p$ is $p-1 = 3\cdot 2^{30}$. If we take $w = 5$ as the primitive element, the multiplicative subgroup of $2^{13}$ elements generator will be $\omega = 5^{3\cdot 2^{17}}$
\end{example}

This kind of subgroups comes with very useful property: for each element in
two-adicity subgroup $\mathbb{H}$, the additive inverse element
can be calculated by a simple equation over the element power.

\begin{proposition}
    Suppose $\mathbb{H} \leq \mathbb{F}_p$ is a subgroup of order $r$ with
    generator $h = w^{(p-1)/r}$. Then, the additive inverse for $x = h^i \in
    \mathbb{H}$ is $h^j$ where $j = i + \frac{r}{2} \pmod{r}$.
\end{proposition}

\textbf{Proof.} The sum of $x$ and $-x$ must equal to zero modulo $p$, so:
\begin{equation*}
    \begin{aligned}
        x + (-x) &= w^{(p-1)i/r} + w^{(p-1)j/r} = w^{(p-1)i/r}(1 + w^{(p-1)(j-i)/r}) \\ &= w^{(p-1)i/r}(1 + w^{(p-1)/2}).
    \end{aligned}
\end{equation*}

Now note that $w^{(p-1)/2} = -1$ which completes the proof. $\blacksquare$

\begin{remark}
The equation $w^{p - 1} = 1$ is obtained from the order property of the
primitive element $w$ in the multiplicative group $\mathbb{F}^{\times}_p$. 
\end{remark}
\begin{remark}
This provides us with an additional important property beyond element's power
computation: when working with a negative element, its power shift equals half
the size of the subgroup so, squaring the elements within this subgroup results
in a smaller subgroup, reduced by a factor of two. Consequently, to compute the
square of the subgroup, it suffices to square only the first half of its
elements (powers $0, 1, 2, 3, \dots, \frac{r}{2}$).
\end{remark}

\subsection{Protocol definition}

\subsubsection{Trace, evaluation domain and commitment}
Now, we are going to prove that some statement holds on the given sequence of elements.

\begin{definition}
We call \textbf{trace} a sequence of elements from $\mathbb{F}$ that represents our witness. This sequence contains private and public values together and follows certain constraints.
\end{definition}

\begin{example}
The \textbf{Fibonacci square sequence} is a sequence of elements defined over
$\mathbb{F}$ as follows: 
\begin{equation*}
a_{j+2} = a_{j+1}^2 + a_{j}^2  
\end{equation*}

Then we can, for example, prove the following statement: \textcolor{blue!60!black}{\textit{I know a field
element $w \in \mathbb{F}$ such that the $k^{\text{th}}$ element of the Fibonacci
square sequence ($a_k$) starting with $x$ and $w$ is $y$.}} Formally,
this can be written as:
\begin{equation*}
    \mathcal{R}_{\text{Fib}} = \left\{ \begin{matrix}
        \textbf{Public Statement:} \; (x, y, k) \\
        \textbf{Witness:} \; w  
    \end{matrix} \;\Big|\;  \begin{matrix}
        a_0 = x, a_1 = w, a_k = y \; \text{with} \\ a_{j+2} = a_{j+1}^2 + a_j^2 \; \text{for all} \; j \in [k] 
    \end{matrix}    
    \right\}
\end{equation*} 

For concreteness, let us take $k=1023$, $x = 1$, and $y=2338775057$.
\end{example}

Following the Unisolvence Theorem, the trace $\{a_j\}_j$ is implied to be an evaluation
of some unknown \textbf{trace polynomial} of degree equal to the length of the 
sequence $\{a_j\}_j$. Also, to be
evaluable on the two-adicity subgroup, the size of the trace has to be a power
of two.

\begin{definition}
We call \textbf{domain} a two-adicity subgroup $\mathbb{G} \leq
\mathbb{F}^{\times}$ where we evaluate our polynomials.
\end{definition}

\begin{example}
In our example, we put trace a sequence $\{a_j\}_j$ of first $1023$ elements of
the Fibonacci square sequence over $\mathbb{F}_p$, where $p=3\cdot 2^{30} + 1$.
\begin{equation*}
1, 1, 2, 5, 29, \ldots
\end{equation*}
To interpolate our trace polynomial we select as a domain a two-adicity subgroup
of $2^{10}$ elements from $\mathbb{F}^\times_p$ with a generator $g =
5^{\frac{3\cdot 2^{30}}{2^{10}}} = 5^{3 \cdot 2^{20}}$ (here $5$ is the
primitive element in the multiplicative group $\mathbb{F}^\times_p$). That being
said, $\mathbb{G} = \{g^i\}_{i \in [1024]}$.
\end{example}

Next, using the Lagrange interpolation over $(g^j, a_j)_{j \in [k]}$ points
we compute a trace polynomial $f \in \mathbb{F}[x]$. Note that the interpolation 
can be done in $O(k\log k)$, as shown in NTT section.

\begin{definition}
We call \textbf{evaluation domain} a two-adicity coset $\mathbb{E} = w\mathbb{H}
\leq \mathbb{F}_p^{\times}$, where $\mathbb{H} \leq \mathbb{F}_p^{\times}$ is a
two-adicity subgroup, that is larger $\rho \in \mathbb{N}$ times (typically a
relatively small constant) than the domain. In other words,
$\text{ord}(\mathbb{H}) = \rho \cdot \text{ord}(\mathbb{G})$.
\end{definition}

\begin{example}
In our case we select a two-adicity subgroup $\mathbb{H}$ of $2^{13}$ elements
from $\mathbb{F}_p^\times$ with $\rho = 8$ as $\mathbb{H} = \{h^i\}_{i \in
[8192]}$ where $h = 5^{3 \cdot 2^{17}}$. Then, we define the \emph{evaluation
domain} as $\mathbb{E}=5\mathbb{H} = \{5h^i\}_{i \in [8192]}$.
\end{example}

We build a Merkle tree over the values $\{f(e)\}_{e \in \mathbb{E}}$ and label
its root as a \textbf{trace polynomial commitment}. This approach will also be
used to commit other polynomials during the protocol walkthrough.

The \textbf{constraints} in STARK protocol are expressed as polynomials
evaluated over the trace cells, which are satisfied if and only if the
computations are correct.

\begin{example}
Obviously, our initial statement consists of the following three requirements:
\begin{enumerate}
    \item The element $a_0$ is equal to $1$;
    \item The element $a_{1022}$ is equal to $2338775057$;
    \item Each element $a_{i+2}$ is equal to $a_{i+1}^2 + a_{i}^2$.
\end{enumerate}
\end{example}
To verify that our committed trace polynomial satisfies all constraints, we can
check that it has corresponding roots. In particular, according to the selected
interpolation points $\{(g^i, a_i)\}_{i \in [k]}$, the relation $r(a_i, a_j) =
0$ can be rewritten as $r(f(g^i), f(g^j)) = 0$.
\begin{example}
For our Fibonacci trace we have the following constraints to be checked over the interpolated polynomial:
\begin{enumerate}
    \item \textit{The element $a_0$ is equal to $1$} translated to: $f(x)-1$ has root at $x = g^0 = 1$;
    \item \textit{The element $a_{1022}$ is equal to $2338775057$} translated to: $f(x) - 2338775057$ has root at $x = g^{1022}$;
    \item \textit{Each element $a_{i+2}$ is equal to $a_{i+1}^2 + a_{i}^2$} translated to: $f(g^2x) - f(gx)^2 - f(x)^2$ has roots in $\mathbb{G} \setminus \{g^{1021}, g^{1022}, g^{1023}\}$
\end{enumerate}
\end{example}

To ensure that the specified polynomials have roots in given values, we can use the following property: if polynomial $f(x) \in \mathbb{F}[x]$ has root in $x_0$ then the $\frac{f(x)}{x - x_0}$ is also a polynomial in $\mathbb{F}[x]$.

\begin{example}
Finally, we define the following STARK constraints:
\vspace{-1mm}
\begin{gather*}
    p_0(x) = \frac{f(x)-1}{x - 1} \\
    p_1(x) = \frac{f(x) - 2338775057}{x - g^{1022}} \\
    p_2(x) = \frac{f(g^2x) - f(gx)^2 - f(x)^2}{\prod_{i=0}^{1020} (x - g^i)}
\end{gather*}
\vspace{-1mm}
Unfortunately, the $p_2$ polynomial still looks inconvenient to work with, so we
may want to simplify it (this is not a part of the protocol in general, but you
always may want to simplify your equations to achieve better proving time). Note
that $p_2$ is \textit{almost} a vanishing polynomial of $\mathbb{G}$, which has
a form $x^{\text{ord}(\mathbb{G})} - 1$, except for points $g^{1021},
g^{1022}, g^{1023}$. In other words, we can simplify the denominator as:
\begin{xequation*}
    \prod_{i=0}^{1020} (x - g^i) = \frac{x^{1024} - 1}{(x-g^{1021})(x-g^{1022})(x-g^{1023})}
\end{xequation*}

Note, that while evaluating our polynomial on a larger domain
then $\mathbb{G}$ we should only ensure that the resulting polynomial still
holds the relation $f(g^i) = a_i$, so it is acceptable to use properties that
only work over $\mathbb{G}$. So, finally we have:
    \begin{xequation}
        p_2(x) = \frac{(f(g^2x) - f(gx)^2 - f(x)^2)(x - g^{2021})(x - g^{2022})(x - g^{2024})}{x^{1024} - 1}  
    \end{xequation}
\end{example}

In addition, there is one obvious requirement for the STARK constraints: the
verifier should be able to compute the constraints polynomials $p_i(x)$ using
only the given trace polynomial evaluations for the certain $x$.

\begin{remark}
In our Fibonacci example, verifier can check the constraint polynomials
evaluation by requesting only $f(x)$, $f(gx)$ and $f(g^2x)$ --- the values
committed in the trace polynomial commitment.
\end{remark}

To combine all our constraints into a single polynomial, we can follow a
commonly used principle by taking a linear combination with the challenges from
the verifier. In particular, after receiving trace polynomial commitment from
the prover, the verifier selects scalars $\alpha_1,\dots,\alpha_m$ and sends it
to the prover. Then, the prover puts the \textbf{composition polynomial} as:
\begin{xequation*}
    \text{CP}(x) := \sum_{j = 1}^m \alpha_j\cdot p_j(x)
\end{xequation*}
Additionally, prover also commits this polynomial by evaluating on the evaluation domain and building a Merkle tree.

\begin{example}
The Fibonacci composition polynomial looks like as follows:
    \begin{gather*}
        \text{CP}(x) = \alpha_0 p_0(x) + \alpha_1 p_1(x) + \alpha_2 p_2(x) =\\
        \alpha_0 \frac{f(x)-1}{x - 1} + \alpha_1 \frac{f(x) - 2338775057}{x - g^{1022}} + \\
        \alpha_2 \frac{(f(g^2x) - f(gx)^2 - f(x)^2)(x - g^{2021})(x - g^{2022})(x - g^{2024})}{x^{1024} - 1}
    \end{gather*}
\end{example}

\subsubsection{FRI protocol}
In general, our goal is to verify that the committed polynomial $\text{CP}(x)$
satisfies all our constraints, by checking it's evaluation at a random point
from the evaluation domain that the verifier selects. Anyway, we can face the
problem when the malicious prover constructs a larger polynomial that accepts
lots of possible roots from our field (even $2^{64}$ field is still insecure for
just checking the evaluation at one point). That is why we have to make sure
that the committed polynomial degree lies in the acceptable range (the upper
bound depends on the trace size).

The final stage of the STARK protocol is a \textbf{Fast Reed-Solomon IOP of Proximity (FRI)}. FRI is a protocol between a prover and a verifier, which establishes that a given evaluation belongs to a polynomial of low-degree. In this context \textit{low} means no more than $\rho$ times bigger than the trace.

The key idea of FRI protocol is to move from a polynomial of degree $n$ to a
polynomial of degree $n/2$ until we get a constant value. Let's consider the
polynomial $z_0(x) = \sum_i a_i\cdot x^i$ of degree $n=2^t$ and the evaluation
domain $\mathbb{E}_0 = \mathbb{E}$. We suppose to group the \textit{odd} and the
\textit{even} coefficients of the $z_0$ together into the two separate
polynomials($z_0^O$ and $z_0^E$ respectively):
\begin{xequation*}
    \begin{aligned}
        z_0^O(x^2) = \sum_{i=0}^{n/2} (a_{2i+1}\cdot x^{2i}), \quad z_0^E(x^2) = \sum_{i=0}^{n/2} (a_{2i}\cdot x^{2i})
\end{aligned}
\end{xequation*}

Or, in a more comfortable form (we have already examined why searching of $-x$
can be done easily in our two-adicity subgroup):
\begin{xequation*}
    \begin{aligned}
        z_0^E(x^2) = \frac{z_0(x) + z_0(-x)}{2}, \quad z_0^O(x^2) = \frac{z_0(x) - z_0(-x)}{2x}
    \end{aligned}
\end{xequation*}

Then, we define a next-layer of the FRI polynomial as $z_1(x^2) = z_0^E(x^2) +
\beta z_0^O(x^2)$, where $\beta$ is a challenge received from verifier. The
next-layer evaluation domain is also simple to compute: $\mathbb{E}_1 =
\{(w\cdot h_i)^2\}_{i \in [\text{ord}(\mathbb{E}_0)/2]}$ as squaring the
other elements in $\mathbb{E}_0$ will result in the same values.

Next, we commit to the $z_1(x^2)$ using a next-layer evaluation domain
$\mathbb{E}_1$ (is also reduced by a factor two) and continue to repeat the
described operations until $z_j(x^{2^j})$ becomes constant.

\vspace{-2mm}

\begin{tcolorbox}[title=Interactive ZK-STARK protocol,
    colback=blue!5!white,
    colframe=blue!75!black,
    colbacktitle=blue!25!white,
    coltitle=blue!20!black,
    fonttitle=\bfseries,
    boxrule=1.25pt,
    subtitle style={boxrule=0pt,
    colback=blue!20!white,
    colupper=blue!75!gray} ]
    \small

    The prover and the verifier run the interactive version of the ZK-STARK
    protocol. Both know the statement to be proved, that is defined by the
    constraint polynomials and the field $\mathbb{F}_p$ to work over. Prover also
    knows the witness to be able to generate the trace.

    \tcbsubtitle{Preparation}
    \begin{itemize}[label=\ding{51}]
        \item The prover interpolates trace polynomial $f(x)$ and submits its
        commitment to the verifier.
        \item The verifier selects challenges random $\alpha_i \in \mathbb{F}_p$ 
        and sends to the prover.
        \item The prover builds the composition polynomial $\text{CP}(x)$ and
        submits its commitment to the verifier.
    \end{itemize}

    \tcbsubtitle{FRI}
    \begin{itemize}[label=\ding{51}]
        \item The verifier selects random $j \in [\text{ord}(\mathbb{E})]$, sets
        $c \gets w\cdot h^j$ and sends it to the prover.
        \item The prover responds with the $\text{CP}(c), \text{CP}(-c)$ and all
        $f(x)$ required to check $\text{CP}$ evaluation with corresponding Merkle
        proofs to them.
        \item The verifier checks Merkle proofs and the evaluation of
        $\text{CP}(c)$ by evaluating the constraints polynomials $p_j(c)$.
        \item The prover and the verifier go through the FRI protocol for
        $z_0(x) = CP(x)$ where the prover commits to the layer-$j$ polynomial
        $z_j(x)$, the verifier selects a challenge $\beta$ and queries from the
        prover $z_j(c), z_j(-c)$ to compute $z_{j+1}(c)$ until $z_k(x), j \leq
        \log_2(\deg \text{CP})$ becomes constant.
    \end{itemize}
    
\end{tcolorbox}

The non-interactive version of the presented protocol can be easily built
obtaining the Fiat-Shamir heuristics.

The soundness of the presented STARK protocol follows from the impossibility to
commit any possible evaluation of the forgery $\text{CP}(x)$ over evaluation
domain $\mathbb{E}$ and simultaneously prove that $\text{CP}(x)$ is a low-degree
polynomial by the FRI protocol. Since the size of $\mathbb{E}$ is $\rho$ times bigger
then the maximum allowed polynomial degree (that directly depends on the size of
the trace), the attacker either can't construct such a polynomial or can't
construct a low-degree polynomial, so a valid low-degree composition polynomial
can only be obtained using a valid trace.

\vspace{-2mm}

\begin{example}
    Finally, let's overview the first steps of the ZK-STARK protocol applied to our Fibonacci example:
    
    \begin{enumerate}
        \item The protocol defines the public constraints such as 2023-th
        element of sequence, field $\mathbb{F}_p$, etc.
        \item The prover generates the trace $a$ where $a_0 = 1, a_1 = 3141592,
        a_i = a_{i-1}^2 + a_{i-2}^2$, evaluates the trace polynomial $f(x)$ over
        the evaluation domain and sends it's commitments to the verifier. 
        \item The verifier selects challenges $\alpha_0, \alpha_1, \alpha_2 \in
        \mathbb{F}$ and shares them with the prover.
        \item The prover evaluates the composition polynomial $\text{CP}(x)$
        over evaluation domain and sends it's commitments to the verifier.
        \item The verifier selects random $i \in [8192-16]$, puts $c = 5\cdot
        h^i$ and sends it to the prover.
        \item The prover responds with the $f(c), f(gc), f(g^2c), \text{CP}(c),
        \text{CP}(-c)$ and corresponding Merkle proofs to them.
        \item The verifier checks Merkle proofs and the evaluation of
        $\text{CP}(c)$ by evaluating the constraint polynomials $p_0(c), p_1(c),
        p_2(c)$.
        \item The prover and the verifier go through the FRI protocol for
        $z_0(x) = \text{CP}(x)$ until $z_i(x), i \in [12]$ becomes constant.
    \end{enumerate}
    
\end{example}

\subsection{Protocol security}
Most of the existing versions of the STARK protocol leverage on several
optimizations to achieve better proving and verification time. The key point
here is that each FRI query check adds $\log_2(\rho)$ bits of security, so we can
skip some of these checks if the security level is already satisfied. One more
optimization is to include a proof-of-work computation into the protocol that
should be done before FRI with dependency on the committed values. It can be
useful because the verification of the proof-of-work is less expensive then the
verification of the FRI step while still increases the computation cost for the
malicious prover. 

More precisely, let's assume that the desired security level of the protocol is
$\lambda$. First of all, we obviously have to use a proper collision-resistant
hash function with $2\lambda$ bits output. Then, according to the StarkWare's
definition of the STARK protocol, the resulting security is defined as follows:
\begin{xequation*}
    \lambda \geq \min\{ \delta + \log_2(\rho) \cdot s, \log_2(|\mathbb{F}|) \} - 1
\end{xequation*}
where $\delta$ -- number of the proof-of-work bits, $s$ -- number of the FRI queries.

\begin{example}
If the protocol is deployed over $256$-bit field and the domain ratio is $\rho =
8$, to achieve the $128$ bit security we can for example execute $33$ FRI query
and evaluate $29$ proof-of-work bits: $\min\{29+3\cdot 33, 256\} = 128$. 
\end{example}

\subsection*{Acknowledgements}

This work was inspired by \href{https://starkware.co/stark-101/}{``STARK-101''}
course by StarkWare and
\href{https://vitalik.eth.limo/general/2017/11/09/starks_part_1.html}{``STARKs''}
series by Vitalik Buterin.

\end{document}
